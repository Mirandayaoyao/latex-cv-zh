\cvsection{研究经历}
\begin{cventries}
  \cventry   
    {指导:台大城乡所所长-张圣琳}
    {}
    {蒲江\& 坪林}
    {2015.7- 2016.1}    
    {
      \begin{cvitems}
      	\item{\textbf{两岸城市与产业规划发展--以四川蒲江和台湾坪林为例}\\ 
        从城市与乡村共同体的角度对于蒲江和坪林这样具有产业经济危机、水资源环境条件限制的农村地区,通过蒲江社会设计和台湾新北坪林的茶文化创意教学基地,从社会面和设计面来共同面对区域发展和产业发展的连结。
        }
      \end{cvitems}
      }
      
  \cventry   
    {}
    {}
    {台北}
    {2013.5 - 2015.7}    
    {
      \begin{cvitems}
        \item{ \textbf{台北机厂文化资产保存} \\
        为了保存和再利用此工业遗产,我们提出了一个再利用构想包括教育、国际交流、社区发展、就业、活动平台等。}
      \end{cvitems}
    }
    
    \cventry   
    {}
    {}
    {台北}
    {2014.9 - 2015.1}    
    {
      \begin{cvitems}
        \item{\textbf{公馆楼改造计划 }\\
        通过家具自力营造和举办活动\& 反馈来改善本学院大楼内部环境。
        }
      \end{cvitems}
    }  
    
    \cventry   
    {指导:南大都市社会研究所所长-张红艳}
    {}
    {南京}
    {2013.5 - 2013.7}    
    {
      \begin{cvitems}
        \item{\textbf{高淳游子山森林公园景区提升规划}\\
         负责童话镇板块、民俗村、美丽乡村示范带、宣传片等方案设计。
        }
      \end{cvitems}
    }
    
    
    
   \cventry   
    {指导: 南大社科所所长-张学为}
    {}
    {南京}
    {2012.5 - 2013.4}    
    {
      \begin{cvitems}
        \item{\textbf{主持南京大学校级创新计划}\\ %%
        作为负责人,带领团队前往苏州某县进行银行揽存的社会学调查,成果为论文《基于特殊关系的银行揽存背后的权力博弈》。
        }
      \end{cvitems}
    }

\end{cventries}


